%\chapter{SMT-RAT classes} \label{AppA}

\addcontentsline{toc}{chapter}{Appendix}
\chapter{Gr\"obner module settings} \label{App:Gbsettings}
We give a small overview of some settings to tune the Gr\"obner module.
\section*{Gr\"obner basis settings}
\begin{itemize}
 \item \small{\begin{verbatim} passGB { true, false } \end{verbatim}} 
 We can set if we would like to pass the Gr\"obner basis or the original equalities. Sometimes, setting this to false is better, as Gr\"obner bases for some inputs grow very large.
 \item \small{\begin{verbatim} getReasonsForInfeasibility { true, false } \end{verbatim}}
 This setting enables the reason vector / extracting the conflict set from the reason vector. There is a very minor overhead when this is enabled, but smaller conflicts do improve the computation quite much.
 \item \small{\begin{verbatim} passWithMinimalReasons { true, false } \end{verbatim}}
 This setting enables the reason vector / extracting the conflict set from the reason vector. There is a very minor overhead when this is enabled, but smaller conflicts do improve the computation quite much.
\item \small{\begin{verbatim} TransformIntoEqualities 
{ ALL_INEQUALIES, ONLY_NONSRICT_EQUALITIES, NO_INEQUALITIES } \end{verbatim}} 

 \item \small{\begin{verbatim} check_inequalities { ALWAYS, AFTER_NEW_GB, NEVER };\end{verbatim}} 
 Here, we can set how to handle inequalities: \verb|NEVER| disables the handling of inequalities. \verb|AFTER_NEW_GB| does the reduction of inequalities only if we have calculated a new Gr\"obner basis. \verb|ALWAYS| also reduces new inequalities in each theory call. 
 \item \small{\begin{verbatim} pass_inequalities { AS_RECEIVED, FULL_REDUCED, FULL_REDUCED_IF } \end{verbatim}}
 If \verb|check_inequalities| is not set to \verb|NEVER|, this sets how the inequalities should be passed. \verb|AS_RECEIVED| passes the inequalities unchanged (only if a inequality is superfluous, it is not passed). \verb|FULL_REDUCED| passes the reduced inequalities, and \verb|FULL_REDUCED_IF| passes them on a condition,  which is described by the \verb|decidePassingPolynomial| function. Reduced inequalities may be bigger then the original, but checking the condition might impose an extra overhead. 
 \item \small{\begin{verbatim} after_firstInfeasibleSubset { RETURN_DIRECTLY,
			PROCEED_CONFLICTANDDEDUCTION, PROCEED_ALLINEQUALITIES } \end{verbatim}}
 If during the check of the inequalities a conflict set is found, this can be returned directly (\verb|RETURN_DIRECTLY|), but we may miss other conflicts, which are detected if we proceed. However, it may be a overhead to update everything, so we might only handle the conflicts and deductions and forget about the rest(\verb|PROCEED_CONFLICTANDDEDUCTION|), or we may just go ahead until we have treated all inequalities (\verb|PROCEED_ALLINEQUALITIES|). 
 \item \small{\begin{verbatim} theory_deductions { NONE, ONLY_INEQUALITIES } \end{verbatim}}           
 With this setting we can enable or disable the theory deductions for inequalities. There is no obvious reason to not do theory deductions, as the overhead is very small.
 \item \small{\begin{verbatim} order { LEX, GRLEX, GRREVLEX } \end{verbatim}}
 Here, the used ordering for the Gr\"obner basis can be set.                                                                      
\end{itemize}
\section*{Real NSS settings}
\begin{itemize}
 \item \small{\begin{verbatim} applyNSS { true, false } \end{verbatim}}
 \item \small{\verb| SDPupperBoundNrVariables | $\mathbb{N}$} \\
 The SDP procedure is only called if the number of variables our SDP has to iterate over is smaller than \verb|SDPupperBoundNrVariables|.
 \item \small{\verb| maxSDPdegree | $\mathbb{N}$} \\
 The monomial iterator iterates through all monomials with a degree smaller than \verb|maxSDPdegree|.
 \item \small{\verb| callSDPAfterNMonomials | $\mathbb{N}$} \\
 After the \verb|callSDPAfterNMonomials| newly added monomials,  the SDP procedure is called. Notice that the SDP procedure is also called if the monomial iterator has reached its limit.
 \item \small{\verb| sternBrocotStartPrecisionOneTo | $\mathbb{N}$} \\
In the first attempt to rationalise the solution, this precision has to be reached for each single entry.
 \item \small{\verb| sternBrocotHigherPrecisionFactor | $\mathbb{N}$} \\
After the rationalisation failed, the precision is increased with this factor.
 \item \small{\verb| sternBrocotHigherPrecisionSteps | $\mathbb{N}$} \\
If the module cannot find a rational solution after raising the precision a total number of \verb|findRationalSolutionSteps| steps, the module gives up.
\end{itemize}

% after_firstInfeasibleSubset { RETURN_DIRECTLY, PROCEED_INFEASIBLEANDDEDUCTION, PROCEED_ALLINEQUALITIES };
% theory_deductions { NONE, ONLY_INEQUALITIES };
% check_inequalities { ALWAYS, AFTER_NEW_GB, NEVER };
% 	
% 	
% 		static const bool								 applyNSS								 = true;
%         static const unsigned                            maxSDPdegree                            = 4;
%         static const unsigned                            SDPupperBoundNrVariables                = 6;
% 		static const unsigned							 callSDPAfterNMonomials					 = 6;
% 		static const unsigned							 sternBrocotStartPrecisionOneTo			 = 80;
% 		static const unsigned							 sternBrocotHigherPrecisionSteps		 = 4;
% 		static const unsigned							 sternBrocotHigherPrecisionFactor		 = 10;
% 
% struct GBSettings
% {
%     typedef GiNaCRA::GradedLexicgraphic              Order;
%     typedef GiNaCRA::MultivariatePolynomialMR<Order> Polynomial;
%     typedef GiNaCRA::MultivariateIdeal<Order>        MultivariateIdeal;
%     typedef GiNaCRA::BaseReductor<Order>             Reductor;
%     typedef smtrat::decidePassingPolynomial	     passPolynomial;
% 
%     static const bool                          passGB;
%     static const bool                          getReasonsForInfeasibility;
%     static const bool                          passWithMinimalReasons;
%     static const check_inequalities            checkInequalities                       = ALWAYS;
%     static const pass_inequalities             passInequalities                        = FULL_REDUCED;
%     static const after_firstInfeasibleSubset         withInfeasibleSubset                    = PROCEED_INFEASIBLEANDDEDUCTION;
%     static const theory_deductions                   addTheoryDeductions                     = ONLY_INEQUALITIES;
%     static const unsigned                            setCheckInequalitiesToBeginAfter        = 0;
%     static const bool                                checkInequalitiesForTrivialSumOfSquares = true;
%     static const bool                                checkEqualitiesForTrivialSumOfSquares   = true;
%     static const bool				     transformIntoEqualities		     = false;
% }
% 
\chapter{Benchmark results} \label{app:Benchmarkresults}