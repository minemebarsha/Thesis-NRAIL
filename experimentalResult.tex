%/////////////////////////////////////////////////
%/
\chapter{Experimental Result}
%/
%/////////////////////////////////////////////////
\label{chap:experimental_result}

%/////////////////////////////////////////////////
%/
\section{Experimental Setup}
%/
%/////////////////////////////////////////////////
We evaluate our implementation by running some benchmarks over different strategies as well as solvers.
We have used QF\_NRA benchmarks from SMT-LIB.
We run the benchmarks on a cluster having multiple processors of 2.10 GHz Intel Xeon Platinum 8160 and 8GB of memory.
We decided to use a timeout of 120 seconds (s) for the experiment.

%/////////////////////////////////////////////////
%/
\section{Results}
%/
%/////////////////////////////////////////////////
We have already described in the section \ref{subsubsec:Module_Settings} that we use different strategies \textit{NRARefinementSolver$1$}, \textit{NRARefinementSolver$2$}, $\dots$, \textit{NRARefinementSolver$21$} which are mentioned as  smtrat$1$, smtrat$2$, $\dots$, smtrat$21$, respectively in this chapter. 
The results for smtrat$1$, smtrat$2$, $\dots$, smtrat$21$ are reported in Table \ref{table:results_our_solvers} where the first column represents the solver's name.
The second and third column represents the number of satisfiable and unsatisfiable instances, respectively with average solving time.
The total number of solved instances with the percentage of the solution is reported in the last column from worst to best.
The detailed settings of each solver are also provided in Table \ref{table:ourSolvers_heuristics_sequences} by a pair of heuristic type and sequence of axiom types (Section \ref{subsubsec:Module_Settings}).\newline

\noindent Table \ref{table:results_our_solvers} shows some interesting aspects.
The solver smtrat4 performs the best, whereas smtrat7 performs the worst.
However, smtrat11 and smtrta3 solve the highest number of satisfiable and unsatisfiable instances, respectively.
It is noticeable that six solvers of heuristic type \textit{ALL} which is marked green performs the best in a row.
On the other hand, the red marked rows highlight all solvers of heuristic type \textit{RANDOM} performing worse than other solvers in a row.
Also, this group of solvers solves less unsatisfiable instances compared to other solvers.
We can see that smtrat7, smtrat14 and smtrat21 share same sequence of axiom types from Table \ref{table:ourSolvers_heuristics_sequences}.
Interestingly, two solvers smtrat7 and smtrat21 among these three solvers rank at the last which demonstrates that sequence $7$ is not effective because it has the most  expensive axiom monotonicity at the very beginning of the sequence.
Sequence $4$ is followed by both our best solver smtrat4 and the highest number of satisfiable instances solving solver smtrat11.
Moreover, smtrat4 and smtrat11 take less solving time on average for both instances than most of the solvers.
So, it can be said that sequence $4$ is the most effective sequence where our defined axiom ICP is placed at the beginning and ICP helps to reach the solution more quickly.
It also can be said that a pair of heuristic type ALL and sequence $4$ helps to improves the performance of a solver.
Therefore, we decide to consider smtrat4  for further evaluation which follows heuristic type ALL to collect unsatisfiable axioms by performing refinement over sequence $4$.\newline

\begin{table}[]
    \caption{Number of solved instances for our solvers}
    \centering
    \begin{tabularx}{\textwidth}{lXrrXrrXrr}
    	\toprule
    	\textbf{Solver}
    	&& \multicolumn{2}{c}{\textbf{SAT}}
    	&& \multicolumn{2}{c}{\textbf{UNSAT}}
    	&& \multicolumn{2}{c}{\textbf{Overall}}
    	\\
    	\midrule
    	\rowcolor{black!20}
    	smtrat7
    	&& 1678 & 2.53~s
    	&& 3806 & 1.88~s
    	&& 5484 & 47.7~\%
    	\\ 
    	\rowcolor{black!20}
    	smtrat21
    	&& 1735 & 3.06~s
    	&& 3759 & 1.91~s
    	&& 5494 & 47.8~\%
    	\\
    	\rowcolor{red!20}
    	smtrat13
    	&& 1867 & 1.12~s
    	&& 3659 & 1.12~s
    	&& 5526 & 48.1~\%
    	\\
    	\rowcolor{red!20}
    	smtrat14
    	&& 1893 & 1.68~s
    	&& 3646 & 1.19~s
    	&& 5539 & 48.2~\%
    	\\
    	\rowcolor{red!20}
    	smtrat9
    	&& 1899 & 1.39~s
    	&& 3659 & 1.30~s
    	&& 5558 & 48.4~\%
    	\\
    	\rowcolor{red!20}
    	smtrat10
    	&& 1895 & 1.54~s
    	&& 3664 & 1.30~s
    	&& 5559 & 48.4~\%
    	\\
    	\rowcolor{red!20}
    	smtrat12
    	&& 1902 & 1.61~s
    	&& 3657 & 1.14~s
    	&& 5559 & 48.4~\%
    	\\
    	\rowcolor{red!20}
    	smtrat11
    	&& 1916 & 1.21~s
    	&& 3663 & 1.17~s
    	&& 5579 & 48.6~\%
    	\\
    	\rowcolor{red!20}
    	smtrat8
    	&& 1914 & 1.39~s
    	&& 3669 & 1.24~s
    	&& 5583 & 48.6~\%
    	\\
    	\rowcolor{blue!20}
    	smtrat16
    	&& 1801 & 1.94~s
    	&& 3818 & 1.84~s
    	&& 5619 & 48.9~\%
    	\\
    	\rowcolor{blue!20}
    	smtrat20
    	&& 1813 & 1.70~s
    	&& 3810 & 1.83~s
    	&& 5623 & 48.9~\%
    	\\
    	\rowcolor{blue!20}
    	smtrat17
    	&& 1809 & 1.99~s
    	&& 3827 & 1.92~s
    	&& 5636 & 49.1~\%
    	\\
    	\rowcolor{blue!20}
    	smtrat19
    	&& 1831 & 1.83~s
    	&& 3816 & 1.83~s
    	&& 5647 & 49.2~\%
    	\\
    	\rowcolor{blue!20}
    	smtrat18
    	&& 1817 & 2.28~s
    	&& 3840 & 2.10~s
    	&& 5657 & 49.2~\%
    	\\
    	\rowcolor{blue!20}
    	smtrat15
    	&& 1859 & 2.34~s
    	&& 3826 & 2.04~s
    	&& 5685 & 49.5~\%
    	\\
    	\rowcolor{green!20}
    	smtrat6
    	&& 1811 & 1.84~s
    	&& 3911 & 1.99~s
    	&& 5722 & 49.8~\%
    	\\
    	\rowcolor{green!20}
    	smtrat5
    	&& 1828 & 1.94~s
    	&& 3897 & 1.76~s
    	&& 5725 & 49.8~\%
    	\\
    	\rowcolor{green!20}
    	smtrat1
    	&& 1820 & 1.79~s
    	&& 3908 & 1.61~s
    	&& 5728 & 49.9~\%
    	\\
     	\rowcolor{green!20}
    	smtrat2
    	&& 1827 & 1.99~s
    	&& 3902 & 1.75~s
    	&& 5729 & 49.9~\%
    	\\
    	\rowcolor{green!20}
    	smtrat3
    	&& 1810 & 1.73~s
    	&& 3924 & 2.17~s
    	&& 5734 & 49.9~\%
    	\\
     	\rowcolor{green!20}
    	smtrat4
    	&& 1849 & 1.64~s
    	&& 3914 & 1.67~s
    	&& 5763 & 50.2~\%
    	\\
    	\bottomrule
    \end{tabularx}
    \label{table:results_our_solvers}
\end{table}

\begin{table}[]
    \centering
    \caption{Settings of our solvers}
    \begin{tabular}{|c|c|c|}
    \hline
    \textbf{Solver}                & \textbf{Heuristic type}                           & \textbf{Sequence no / Sequence of axiom types}                                                       \\ \hline
    smtrat1                        & \multirow{7}{*}{ALL}                         & 1 / zero, tangent plane, ICP, congruence, monotonicity                      \\ \cline{1-1} \cline{3-3} 
    smtrat2                        &                                              & 2 / tangent plane, zero, ICP, congruence, monotonicity                      \\ \cline{1-1} \cline{3-3} 
    smtrat3                        &                                              & 3 / tangent plane, ICP, zero, congruence, monotonicity                      \\ \cline{1-1} \cline{3-3} 
    smtrat4                        &                                              & 4 / ICP, zero, tangent plane, congruence, monotonicity                      \\ \cline{1-1} \cline{3-3} 
    smtrat5                        &                                              & 5 / ICP, tangent plane, zero, congruence, monotonicity                      \\ \cline{1-1} \cline{3-3} 
    smtrat6                        &                                              & 6 / congruence, zero, tangent plane, ICP, monotonicity                      \\ \cline{1-1} \cline{3-3} 
    smtrat7                        &                                              & 7 / monotonicity, zero, tangent plane, ICP, congruence                      \\ \hline
    \multicolumn{1}{|l|}{smtrat8}  & \multicolumn{1}{c|}{\multirow{7}{*}{RANDOM}} & \multicolumn{1}{l|}{1 / zero, tangent plane, ICP, congruence, monotonicity} \\ \cline{1-1} \cline{3-3} 
    \multicolumn{1}{|l|}{smtrat9}  & \multicolumn{1}{l|}{}                        & \multicolumn{1}{l|}{ 2 / tangent plane, zero, ICP, congruence, monotonicity} \\ \cline{1-1} \cline{3-3} 
    \multicolumn{1}{|l|}{smtrat10} & \multicolumn{1}{l|}{}                        & \multicolumn{1}{l|}{3 / tangent plane, ICP, zero, congruence, monotonicity} \\ \cline{1-1} \cline{3-3} 
    \multicolumn{1}{|l|}{smtrat11} & \multicolumn{1}{l|}{}                        & \multicolumn{1}{l|}{4 / ICP, zero, tangent plane, congruence, monotonicity} \\ \cline{1-1} \cline{3-3} 
    \multicolumn{1}{|l|}{smtrat12} & \multicolumn{1}{l|}{}                        & \multicolumn{1}{l|}{5 / ICP, tangent plane, zero, congruence, monotonicity} \\ \cline{1-1} \cline{3-3} 
    \multicolumn{1}{|l|}{smtrat13} & \multicolumn{1}{l|}{}                        & \multicolumn{1}{l|}{6 / congruence, zero, tangent plane, ICP, monotonicity} \\ \cline{1-1} \cline{3-3} 
    \multicolumn{1}{|l|}{smtrat14} & \multicolumn{1}{l|}{}                        & \multicolumn{1}{l|}{7 / monotonicity, zero, tangent plane, ICP, congruence} \\ \hline
    smtrat15                       & \multirow{7}{*}{FIRST}                       & 1 / zero, tangent plane, ICP, congruence, monotonicity                      \\ \cline{1-1} \cline{3-3} 
    \multicolumn{1}{|l|}{smtrat16} &                                              & \multicolumn{1}{l|}{ 2 / tangent plane, zero, ICP, congruence, monotonicity} \\ \cline{1-1} \cline{3-3} 
    smtrat17                       &                                              & 3 / tangent plane, ICP, zero, congruence, monotonicity                      \\ \cline{1-1} \cline{3-3} 
    smtrat18                       &                                              & 4 / ICP, zero, tangent plane, congruence, monotonicity                      \\ \cline{1-1} \cline{3-3} 
    smtrat19                       &                                              & 5 / ICP, tangent plane, zero, congruence, monotonicity                      \\ \cline{1-1} \cline{3-3} 
    smtrat20                       &                                              & 6 / congruence, zero, tangent plane, ICP, monotonicity                      \\ \cline{1-1} \cline{3-3} 
    \multicolumn{1}{|l|}{smtrat21} &                                              & \multicolumn{1}{l|}{7 / monotonicity, zero, tangent plane, ICP, congruence} \\ \hline
    \end{tabular}
    \label{table:ourSolvers_heuristics_sequences}
\end{table}

\noindent Table \ref{table:smtrat4_vs_mathsatAndZ3} has the same structure as Table \ref{table:results_our_solvers}.
We compared smtrat4 with other two SMT solvers z3 and mathsat.
It is clearly visible that z3 and mathsat have excellent performance than smtrat4 in all cases.
The SMT solver z3 implements expensive and complete techniques based on variants of cylindrical algebraic decomposition \cite{Cimatti:2018:ILS:3274693.3230639}.
The SMT solver mathsat implements the incremental approach as described in \cite{Cimatti:2018:ILS:3274693.3230639} which we have followed in this thesis.
So, we are interested to compare smtrat4 with mathsat as both implements the same technique.\newline

\noindent The solver smtrat4 solves around $2$K satisfiable and $4$K unsatisfiable instances, whereas mathsat solves slightly above $3.5$K satisfiable and $5$K unsatisfiable instances.
So, smtrat4 performs much better for unsatisfiable benchmarks than satisfiable benchmarks.
For satisfiable benchmarks, the performance of smtrat4 is also very unsatisfactory compared to mathsat.
Our solver outputs only UNSAT by the SMT solver and SAT by checking if input NRA formula $\varphi$ is satisfied by the estimated model $\mu$ (Figure \ref{fig:system_architecture_ours}).
Remember that this $\mu$ was being created by guessing the value of original variables (containing in $\varphi$) zero if and only if the solution to the linearization problem does not provide assignments to all original variables.
Hence, it is evident that $\mu$ does not contain the solutions for most of the satisfiable benchmarks and as a result of wrong guessing, smtrat4 results in less satisfiable instances.
We may reach the solutions for satisfiable benchmarks after some loops which can cause a timeout.
So, this delay is also a reason for getting lots of timeouts.
On the contrary, mathsat finds out the actual solutions for satisfiable benchmarks and solves a satisfactory number of satisfiable instances.
Moreover, smtrat4 can also be more shiny on unsatisfiable benchmarks if we would have tried to create correct $\mu$.\newline

\begin{table}[!ht]
    \caption{Comparison of smtrat4 with mathsat and z3}    
    \begin{tabularx}{\textwidth}{lXrrXrrXrr}
    	\toprule
    	\textbf{Solver}
    	&& \multicolumn{2}{c}{\textbf{SAT}}
    	&& \multicolumn{2}{c}{\textbf{UNSAT}}
    	&& \multicolumn{2}{c}{\textbf{Overall}}
    	\\
    	\midrule
    	smtrat4
    	&& 1849 & 1.64~s
    	&& 3914 & 1.67~s
    	&& 5763 & 50.2~\%
    	\\
    	mathsat
    	&& 3560 & 1.20~s
    	&& 5286 & 1.80~s
    	&& 8846 & 76.9~\%
    	\\
    	z3
    	&& 5004 & 2.14~s
    	&& 5099 & 1.63~s
    	&& 10103 & 87.9~\%
    	\\
    	\bottomrule
    \end{tabularx}
    \label{table:smtrat4_vs_mathsatAndZ3}
\end{table}

\begin{figure}
\caption{Survival plots smtrat4 with mathsat and z3} 
\begin{tikzpicture}
	\begin{axis}[
			xlabel=\# of solved instances,
			ylabel=runtime (s),
			ymode=log,
			minor y tick num=1,
			xmin=-200,
			xmax=12000,
			ymin=0,
			ymax=1200,
			xticklabel=$\pgfmathprintnumber{\tick}$k,
			scaled ticks=false,
			scaled x ticks=manual:{}{\pgfmathparse{#1/1000.0}},
			xtick={0,2000,4000,6000,8000,10000,11447},
			height=5.5cm,
			width=0.9\linewidth,
			legend pos = north west,
			legend cell align = left,
			ymajorgrids = true,
	]

	\addplot[color=red, solid] table[x index=0,y index=1] {experiments/output/plot-z3.data};
	\addplot[color=green, solid] table[x index=0,y index=1] {experiments/output/plot-mathsat.data};
	\addplot[color=blue, solid] table[x index=0,y index=1] {experiments/output/plot-smtrat_4.data};
% 	\addplot[color=black, solid] table[x index=0,y index=1] {experiments/output/plot-smtrat_24.data};
%     \legend{z3, matsat, smtrat4, smtrat24}
    \legend{z3, matsat, smtrat4}
	\end{axis}
\end{tikzpicture}
\label{fig:Survival_plots_smtrat4} 
\end{figure}

\noindent Figure \ref{fig:Survival_plots_smtrat4} illustrates the survival plots for z3, mathsat and smtrat4.
The x-axis shows the number of instances solved within the corresponding time and the x-axis shows the solving time in second.
The survival plots behave exponentially.
The solver smtrat4 and mathsat solve almost $6$K and $9$K instances in total.
However, z3 solves above $10$K in total.
Initially, smtrat4 starts solving instances earlier than z3 followed by mathsat.
The performance of smtrat4 decreases later due to wrong assumption about which is argued above.\newline

\noindent We already know that we have defined three heuristic types and seven sequences of axiom types.
It is argued above that the a pair of heuristic type ALL and sequence $4$ can enhance the performance of a solver.
In other words, ICP has an influence on the enhancement of the performance.
That is why we think about to play around with the sequence by concentrating on ICP the highest.
So, for experimental purpose we create four additional SMT-RAT solvers (smtrat22, $\dots$, smtrat25) with the following sequences of axiom types but with the same heuristic type ALL:

\begin{itemize}
    \item smtrat22 -> ICP, zero, tangent plane, congruence
    \item smtrat23 -> ICP, tangent plane, zero, congruence
    \item smtrat24 -> ICP, zero, ICP, tangent plane, ICP, congruence
    \item smtrat25 -> ICP, tangent plane, ICP, zero, ICP, congruence
\end{itemize}

\begin{table}[h]
\caption{Summary of smrat4 and other additional smtrat solvers}
\begin{tabularx}{\textwidth}{lXrrXrrXrr}
	\toprule
	\textbf{Solver}
	&& \multicolumn{2}{c}{\textbf{SAT}}
	&& \multicolumn{2}{c}{\textbf{UNSAT}}
	&& \multicolumn{2}{c}{\textbf{Overall}}
	\\
	\midrule
	smtrat4
	&& 1849 & 1.64~s
	&& 3914 & 1.67~s
	&& 5763 & 50.2~\%
	\\
	smtrat23
	&& 1948 & 0.59~s
	&& 4055 & 1.72~s
	&& 6003 & 52.2~\%
	\\
	smtrat22
	&& 1998 & 1.07~s
	&& 4035 & 1.63~s
	&& 6033 & 52.5~\%
	\\
	smtrat25
	&& 1956 & 1.10~s
	&& 4108 & 1.83~s
	&& 6064 & 52.8~\%
	\\
	smtrat24
	&& 2003 & 0.97~s
	&& 4100 & 1.57~s
	&& 6103 & 53.1~\%
	\\
	\bottomrule
\end{tabularx}
\label{table:Summary_of_smrat4_and_other_additional_smtrat_solvers}
\end{table}

\noindent We exclude the axiom type monotonicity from the sequence for each of these solvers as monotonicity is the most costly to generate.
The results are reported in Table \ref{table:Summary_of_smrat4_and_other_additional_smtrat_solvers} which maintains the same structure as other tables.
It is highly visible that the total number of solved instances has increased by $3\%$.
The solver smtrat24 solves more than $2$K satisfiable and $4$K unsatisfiable instances  which are less than $2$K and $4$K, respectively for smtrat4.
Notice that smtrat24 follows almost the pattern of the sequence $4$ except that ICP is inserted after each different axiom type in its sequence.
The average time for solving satisfiable and unsatisfiable instances are also impressive.
\newline

