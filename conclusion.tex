%/////////////////////////////////////////////////
%/
\chapter{Conclusion and Future Work}
%/
%/////////////////////////////////////////////////
\label{chap:conclusion_and_Future_Work}
%/////////////////////////////////////////////////
%/
\section{Conclusion}
%/
%/////////////////////////////////////////////////
\label{sec:Conclusion}
his thesis work focuses on how we can decide the satisfiability of logical formulas.
Solving NRA formula is costly compared to LRA formula.
For the ease of solving NRA formula, we decide to perform abstraction for NRA formula.
This thesis work is motivated by the Ph.D. thesis \cite{Cimatti:2018:ILS:3274693.3230639} and the idea is to create a LRA formula for the input NRA formula by abstracting multiplication terms iteratively.
If the LRA model does not satisfy the NRA formula, then refinement is performed over a set of axioms to remove the spurious assignments until we get SAT or UNSAT or UNKNOWN.
Moreover, we have introduced three types of heuristics with different sequences of axioms for collecting unsatisfiable formulas which are added to LRA formula.
We have also introduced ICP axioms for refinement which was one of the future work of \cite{Cimatti:2018:ILS:3274693.3230639}.\newline

\noindent We have integrated this method as a module in SMT-RAT.
We have evaluated the performance of the module in Chapter \ref{chap:experimental_result}.
We have seen that the module can solve $60\%$ instances in total.
For now, the percentage is satisfactory as this module is a prototype and there are scopes available to improve the module which is mentioned in the following section.
We are unable to implement these as we have time limitations.
We are expecting a significant enhancement of the performance once the mentioned future works will be integrating with the module.

\section{Future work}
%/
%/////////////////////////////////////////////////
\label{sec:Future_Work}
% checked
For the time limitation, we could not complete some essential tasks which are needed to make the solver more stable.  
In the future, we need more time to invest in the following areas:
\begin{itemize}
    \item We have already several axioms, but there are still further possibilities to add axioms.
    Currently, we can exclude only boxes by ICP.
    However, it is also possible to exclude the convexes of different shapes based on templates.
    That means we can try other forms of areas which we want to exclude and then derive axioms for them.
    \item So far, we have assigned the values of original variables by guessing if the linearized model $\hat{\mu}$ does not have solutions for all original variables which may results in UNSAT for satisfiable benchmarks.
    So, it is essential to generate solutions to original variables as well as other variables.
    For example, consider the following formula:
    $$\varphi := x^4y^2 - y > 5$$
   The monomial $x^4y^2$ is linearized as follows and we get the linearized formula $\hat{\varphi} := z_4 -y > 5$:
    $$\underbrace{ \underbrace{ \underbrace{ x \ast x }\limits_{z_{1}} \ast \underbrace{ x \ast x }\limits_{z_{1}}}\limits_{z_{2}} \ast \underbrace{ y \ast y }\limits_{z_{3}}}\limits_{z_{4}}$$
    Now, $\hat{\varphi}$ is passed to the SMT LRA solver and the solver outputs SAT with linearized model $\hat{\mu} := \{z_4 = 16, y = 1.414\}$.
    Notice that $\hat{\varphi}$ does not cover all original variables. 
    Currently, we extend $\hat{\mu}$ by guessing zero for $x$ as well as other variables $z_1, z_2$ and $z_3$ which will inevitably lead us to the refinement process.
    Hence, instead of assuming values we can find out the solutions for these variables based on $\hat{\mu}$ (if available):
    \begin{itemize} 
        \item  $z_4 = z_2 \ast z_3 = z_2 \ast y \ast y$.
        So, $z_2 = 4$.
        \item $z_1 = 2$ as $z_2 = z_1^2$.
        \item Finally, $x = 4$.
     \end{itemize}
     In the future, we will integrate this feature to our solver which will resist going through the refinement process unnecessarily for satisfiable benchmarks.
     Most importantly this feature will enhance the solver's performance by solving more satisfiable instances.
\end{itemize}

