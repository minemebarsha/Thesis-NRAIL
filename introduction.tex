%/////////////////////////////////////////////////
%/
\chapter{Introduction}
%/
%/////////////////////////////////////////////////
\label{chap:Introduction}
%/////////////////////////////////////////////////
%/
This thesis is about checking the satisfiability of logical formulas.
If the logical formula is a non-linear formula then it is computationally very expensive.
The idea in the thesis of \cite{Cimatti:2018:ILS:3274693.3230639} is to apply linear approximations which are over approximative.
For non-linear formula, the authors of \cite{Cimatti:2018:ILS:3274693.3230639} define a linear formula whoose solution set over approximates the set of satisfying solutions.
If a solution for the linear formula is found in the over approximation which does not satisfy the non-linear one then they refine the formula by making the solution set smaller.
The solution set gets smaller means it is getting closer to the non-linear formula though in general it is never reached.
That is why, this is incomplete.
In this way, they strict the solution set until the refinement is done.
So, the authors of \cite{Cimatti:2018:ILS:3274693.3230639} proposed an approach, referred to as Incremental Linearization, that trades the use of expensive solvers for non-linear
formula for an abstraction-refinement loop on top of much less expensive solvers for linear formula \cite{Cimatti:2018:ILS:3274693.3230639}.\newline

\noindent Our contribution in this thesis is the implementation of this method in SMT-RAT and following up on heuristic aspects which are not addressed in the PhD thesis \cite{Cimatti:2018:ILS:3274693.3230639}.
We have also contributed in this thesis work by introducing a new axiom referred as Interval Constraint Propagation (ICP) axiom which is one of the axiom types based on which the refinement is performed.
This ICP axiom is not also mentioned in the PhD thesis \cite{Cimatti:2018:ILS:3274693.3230639}.\newline

\noindent The organization of this thesis paper is as follows.
In Chapter \ref{chap:Preliminaries} we introduce basic terms and definitions including a brief explanation on Incremental Linearization for non-linear formulas as in the PhD thesis \cite{Cimatti:2018:ILS:3274693.3230639} which are needed to read this paper.
Chapter \ref{chap:Incremental_Linearization_For_Real_Arithmetic} provides an explanation of our contributions with some algorithms and examples in details.
Chapter \ref{chap:implementation} describes the important parts of implementation.
Chapter \ref{chap:experimental_result} discusses experimental results.
Finally, we conclude the thesis in Chapter \ref{chap:conclusion_and_Future_Work} by highlighting the future work.

